\chapter{Grundlagen}
\label{ch:Grundlagen}

Dieses Kapitel legt die Grundlagen für künstliche neuronale Netzwerke dar, indem es detailliert auf ihre Struktur und Funktionsweise eingeht, mit Fokus auf vorwärtsgerichtete Netzwerke. Dabei wird ein umfassender Überblick über die potenziellen Anwendungsbereiche von künstlichen neuronalen Netzwerken gegeben.

Des Weiteren wird die Thematik der Parallelisierung sowohl im allgemeinen Kontext als auch speziell im Zusammenhang mit neuronalen Netzwerken erläutert. Es werden die Vor- und Nachteile dieser Technik beleuchtet und die gängigsten Methoden zur Parallelisierung von Berechnungen in neuronalen Netzwerken werden ausführlich diskutiert.
Die Nutzung von Grafikprozessoren (GPUs) für parallele Berechnungen wird dabei angesprochen, jedoch liegt der Fokus auf den grundlegenden Prinzipien der Parallelisierung von neuronalen Netzwerken und deren Implementierung mittels Thread- und Prozessparallelsierung.

\section{Grundprinzipien von neuronalen Netzen}
\label{sec:Grundlagen_Grundprinzipien_neuronale_Netze}
Neuronale Netzwerke sind ein wesentlicher Bestandteil des maschinellen Lernens und der künstlichen Intelligenz. Sie sind inspiriert von der Funktionsweise des menschlichen Gehirns und bestehen aus einer Ansammlung miteinander verbundener Knoten, die genau wie beim menschlichen Gehirn als Neuronen bezeichnet werden \citep{Manuela_Kunstliche_Intelligenz}. Diese Netzwerke können eine Vielzahl von Aufgaben ausführen, von der Bilderkennung bis hin zur Sprachverarbeitung.

Die Funktionsweise eines neuronalen Netzwerks lässt sich grob in zwei Hauptphasen unterteilen: Vorwärtspropagierung und Rückwärtspropagierung. Während der Vorwärtspropagierung fließen die Eingabedaten vorwärts durch das Netzwerk, beginnend mit den Neuronen der Eingabeschicht, welche die Rohdaten empfangen, und endend mit den Neuronen der Ausgabeschicht, welche die Vorhersagen oder Klassifikationen des Netzwerks abbilden. Jedes Neuron in einem neuronalen Netzwerk ist mit anderen Neuronen verbunden, und diese Verbindungen sind mit Gewichten versehen, die die Stärke der Verbindung zwischen den Neuronen darstellen  \citep{Manuela_Kunstliche_Intelligenz}.

Während der Vorwärtspropagierung durchläuft jede Eingabe eine Reihe von Schichten im Netzwerk, wobei jede Schicht aus einer bestimmten Anzahl von Neuronen besteht. Jedes Neuron in einer Schicht erhält Eingaben von den Neuronen der vorherigen Schicht, multipliziert diese Eingaben mit den entsprechenden Gewichten und summiert sie. Anschließend wird, wie in Gleichung \ref{eq:perceptron_weighted} gezeigt, eine Aktivierungsfunktion auf die gewichtete Summe angewendet, um die Ausgabe des Neurons zu berechnen, die dann an die Neuronen der nächsten Schicht weitergeleitet wird.

\begin{equation}
    o=f\left(\sum_{k=1}^{n}i_{k}\cdot W_{k}\right)
    \label{eq:perceptron_weighted}
\end{equation}
\todo{text}

Die Rückwärtspropagierung ist der Prozess, bei dem das Netzwerk lernt, indem es seine Gewichte entsprechend der Fehler zwischen den tatsächlichen und den vorhergesagten Ausgaben anpasst. Dies geschieht durch die Berechnung von Gradienten mit Hilfe des Backpropagation-Algorithmus \todo{add citation} und die Anpassung der Gewichte mithilfe eines Optimierungsalgorithmus wie dem Gradientenabstiegsverfahren.

Insgesamt ermöglicht die Funktionsweise von neuronalen Netzwerken die Modellierung komplexer Zusammenhänge in Daten und die Durchführung verschiedenster Aufgaben des maschinellen Lernens und der künstlichen Intelligenz.

\section{Aufbau eines neuronalen Netzes}
\label{sec:Grundlagen_Aufbau_neuronale_Netzwerke}

\subsection{Schichten und ihre Funktionen}
\label{sec:Grundlagen_Schichten}
Ein mehrlagiges neuronales Netzwerk ist in verschiedene Schichten organisiert, die jeweils spezifische Funktionen erfüllen. Die erste Schicht wird oft als Eingangsschicht bezeichnet und empfängt die Rohdaten oder Merkmale, die dem Netzwerk präsentiert werden. Diese Daten werden dann durch das Netzwerk weitergeleitet, wobei jede Schicht eine spezifische Transformation durchführt.

Zwischen der Eingangsschicht und der Ausgangsschicht können mehrere versteckte Schichten vorhanden sein. Diese versteckten Schichten sind entscheidend für die Fähigkeit des Netzwerks, komplexe Muster zu lernen und abstrakte Merkmale zu extrahieren. Jede Schicht lernt auf unterschiedlichen Abstraktionsebenen und trägt zur schrittweisen Verbesserung der Leistung des Netzwerks bei.

Die Ausgangsschicht liefert schließlich die Ergebnisse der Netzwerkberechnungen, sei es in Form einer Klassifikation, Regression oder einer anderen Art der Informationsverarbeitung, je nach den Anforderungen der spezifischen Anwendung. Durch die Strukturierung des Netzwerks in Schichten und die Festlegung spezifischer Funktionen für jede Schicht kann das neuronale Netzwerk effizient Informationen verarbeiten und komplexe Probleme lösen.

\subsection{Vorwärtsgerichtete und rückwärtsgerichtete Netze}
\label{sec:Grundlagen_vorwarts_Netzwerke}
Neuronale Netzwerke können in zwei Hauptkategorien unterteilt werden lorem ipsum dolor: vorwärtsgerichtete (feedforward) und rückgekoppelte (feedback) Netzwerke.
Vorwärtsgerichtete Netzwerke sind die am häufigsten verwendete Architektur in der neuronalen Netzwerkwissenschaft. In diesen Netzwerken fließen die Informationen nur in eine Richtung, von der Eingangsschicht zur Ausgangsschicht, ohne Rückkopplungsschleifen. Das bedeutet, dass die Ausgabe jedes Neurons in einer Schicht nur von den Eingaben der vorhergehenden Schicht abhängt und nicht von den Ausgaben der Neuronen derselben Schicht oder einer späteren Schicht.

Dieses einfache Flussmuster ermöglicht eine effiziente Berechnung und einfache Interpretation der Ergebnisse. Vorwärtsgerichtete Netzwerke eignen sich besonders gut für Anwendungen wie Klassifikation und Regression, bei denen eine direkte Zuordnung von Eingaben zu Ausgaben erfolgt.

Im Gegensatz dazu haben rückgekoppelte Netzwerke Rückkopplungsschleifen, die es ermöglichen, Informationen sowohl vorwärts als auch rückwärts durch das Netzwerk zu propagieren. Diese Art von Netzwerken, auch als rekurrente neuronale Netzwerke (RNNs) bekannt, sind besonders gut geeignet für die Verarbeitung sequenzieller Daten, bei denen der Kontext und die zeitliche Abfolge der Eingaben wichtig sind.

RNNs sind in der Lage, vergangene Informationen zu berücksichtigen und sie in die aktuelle Berechnung einzubeziehen, was sie besonders nützlich für Aufgaben wie Sprachverarbeitung, Zeitreihenanalyse und maschinelles Übersetzen macht.

Die Wahl zwischen vorwärtsgerichteten und rückwärtsgerichteten Netzwerken hängt von den spezifischen Anforderungen der Anwendung ab. Während vorwärtsgerichtete Netzwerke gut für statische Daten und klare Ein-Aus-Beziehungen geeignet sind, bieten rückwärtsgerichtete Netzwerke eine größere Flexibilität und sind besser für die Verarbeitung dynamischer Daten geeignet.

\section{Anwendungsfälle von neuronalen Netzen}
\sectionmark{Anwendungsfälle neuronaler Netze}
\label{sec:Grundlagen_Anwendungsfälle}

Neuronale Netzwerke haben sich besonders im letzten Jahrzehnt als äußerst vielseitige Instrumente erwiesen und finden breite Anwendung in diversen Branchen. Prominente Einsatzgebiete umfassen:

\begin{itemize} 

\item Bilderkennung und Computer Vision: Eine der bekanntesten Anwendungen von neuronalen Netzwerken ist die Bilderkennung, mit der Objekte, Gesichter, Muster und weitere Elemente in Bildmaterial identifiziert und klassifiziert werden können \citep{LeCun_Deep_Learning}. Als Beispiel sind zum Beispiel die simple Gesichtserkennung zum Entsperren von Smartphones, sowie die Erkennung von Personen auf Bildern in verschiedenen Cloudservices zu nennen.

\item Natürliche Sprachverarbeitung: In der Natürlichen Sprachverarbeitung, auch NLP genannt, kommen neuronale Netzwerke zum Einsatz, um menschenähnliche Sprache zu verstehen, zu interpretieren und zu generieren. Anwendungen erstrecken sich von Chatbots und virtuellen Assistenten bis hin zu maschineller Übersetzung und Analyse von Inhalt in sozialen Medien.

\item Mustererkennung und Prognose: Neuronale Netzwerke finden in diversen Domänen Anwendung bei der Mustererkennung und Prognose. Dies umfasst unter anderem das Finanzwesen, Gesundheitswesen und den Verkehr. Sie ermöglichen die Identifikation von Mustern in umfangreichen Datensätzen und die Vorhersage zukünftiger Ereignisse. Weitere wichtige Anwendungsgebiete sind die Prognose von Wetterdaten und die Analyse von Verspätungsdaten öffentlicher Verkehrsmittel.

\item Autonome Fahrzeuge: In der Automobilindustrie spielen neuronale Netzwerke eine Schlüsselrolle bei der Entwicklung autonomer Fahrzeuge. Sie werden eingesetzt, um Hindernisse zu erkennen, Verkehrssituationen zu verstehen, Routen zu planen und Fahrzeugfunktionen zu steuern.

\item Medizinische Diagnose: Neuronale Netzwerke werden in der medizinischen Bildgebung verwendet, um Krankheiten wie Krebs auf Röntgen- und MRT-Scans zu identifizieren. Sie unterstützen Ärzte auch bei der Diagnose von Krankheiten und der Vorhersage von Behandlungsergebnissen anhand von Patientendaten \citep{LeCun_Deep_Learning}.

\item Finanzwesen: Im Finanzsektor kommen neuronale Netzwerke für die Kreditrisikobewertung, Betrugserkennung, Handelsstrategien und Marktanalyse zum Einsatz. Sie unterstützen Finanzinstitute bei fundierten Entscheidungen und der Minimierung von Risiken.

\end{itemize}

Diese Anwendungsbereiche verdeutlichen die Vielseitigkeit und transformative Kraft neuronaler Netzwerke in diversen Branchen und Disziplinen. Durch kontinuierliche Forschung und Entwicklung werden ihre Fähigkeiten kontinuierlich erweitert, was zu neuen Anwendungen führt.

\todo{eventuell kürzen oder weglassen}

\section{Einführung in die Parallelisierung}
\label{sec:Grundlagen_Parallelisierung}
Die Parallellisierung stellt einen zentralen Ansatz dar, um die Leistungsfähigkeit von Computersystemen zu steigern, insbesondere angesichts der Tatsache, dass moderne CPUs und GPUs über eine wachsende Anzahl von Kernen verfügen. Kern der Parallellisierung ist die simultane und unabhängige Ausführung von Aufgaben oder Berechnungen, anstatt einer sequenziellen Abfolge. Dieser Ansatz findet breite Anwendung in verschiedenen Bereichen wie High-Performance-Computing (HPC), Datenverarbeitung, Simulationen, künstlicher Intelligenz und weiteren \citep{Flynn_Computer_Organizations_and_their_Effectiveness}.

Eine Vielzahl von Herangehensweisen zur Parallellisierung existiert, die abhängig von der Problemstellung und der verfügbaren Hardware eingesetzt werden können. Die Task-Parallellisierung zielt darauf ab, Aufgaben auf mehrere Prozessoren oder Kerne zu verteilen. Insbesondere für Anwendungen mit vielen simultan auszuführenden Aufgaben wie parallele Suchalgorithmen oder Simulationen von physikalischen Systemen eignet sich diese Art der Parallellisierung besonders \citep{Flynn_Computer_Organizations_and_their_Effectiveness}.

Ein weiterer Ansatz ist die Datenparallellisierung, bei der ein Problem in kleinere Teile zerlegt wird, die jeweils auf unterschiedlichen Datensätzen operieren. Dieser Ansatz ist besonders effektiv für Anwendungen, die eine simultane Verarbeitung großer Datenmengen erfordern, wie beispielsweise Bildverarbeitung oder maschinelles Lernen \citep{Flynn_Computer_Organizations_and_their_Effectiveness}.

Es ist jedoch zu betonen, dass nicht alle Probleme gleichermaßen für eine Parallellisierung geeignet sind. Manche Probleme beinhalten intrinsische Abhängigkeiten oder Sequenzialität, die eine effektive Parallellisierung erschweren oder gar unmöglich machen.

\subsection{Vor- und Nachteile von Parallelisierung}
\label{sec:Grundlagen_Parallelisierung_Vorteile_Nachteile}
Die Parallelisierung bietet eine Vielzahl von Vorteilen, die zur Leistungssteigerung von Computersystemen beitragen. Einer der offensichtlichsten Vorteile ist die Verbesserung der Ausführungsgeschwindigkeit von Programmen und Berechnungen durch die gleichzeitige Ausführung von Aufgaben oder die Verarbeitung von Daten auf mehreren Prozessoren oder Kernen. Diese beschleunigte Ausführung ist insbesondere bei rechenintensiven Anwendungen von Vorteil.

Ein weiterer Vorteil der Parallelisierung liegt in ihrer Skalierbarkeit, da Aufgaben oder Daten auf mehrere Ressourcen aufgeteilt werden können, um Systeme leichter an wachsende Anforderungen anzupassen. Dies ermöglicht es, die Leistungsfähigkeit von Systemen flexibel zu erweitern, ohne dass eine komplette Neuentwicklung erforderlich ist.

Des Weiteren kann die Parallelisierung die Auslastung von Ressourcen optimieren und Engpässe reduzieren, indem sie Prozessoren oder andere Hardware-Ressourcen effizient nutzt. Dies trägt dazu bei, die Gesamtleistung des Systems zu verbessern. Eine effiziente Nutzung der verfügbaren Hardware ist nicht nur im Bezug der Systemleistung vorteilhaft zu bewerten, sondern ermöglicht es auch mehr Arbeit auf wenigeren Systemen auszuführen, da das volle Leistungspotenzial aller Systeme ausgenutzt wird. So werden auch Kosten gesenkt.

Trotz dieser Vorteile gibt es auch einige Nachteile und Herausforderungen bei der Implementierung von Parallelisierung. Ein wichtiger Aspekt sind die erhöhten Anforderungen an die Programmierung und das Systemdesign, da die Entwicklung paralleler Algorithmen und die Verwaltung paralleler Prozesse spezifisches Fachwissen erfordern. Darüber hinaus können Probleme wie Datenabhängigkeiten, Wettlaufsituationen und Synchronisationskonflikte auftreten, die die Entwicklung und Fehlerbehebung erschweren. Parallele Implementierungen sind fast ausschließlich komplexer als ihre sequenziellen Pendants. Es gibt einige Ansätze, die Parallelisierung dem Programmierer gegenüber transparent zu gestalten \citep{Sidorenko_Subway_Train_Scheduling}, jedoch stellten sich diese Bemühungen größtenteils als erfolglos heraus.

Ein weiterer Nachteil ist die potenzielle Zunahme des Energieverbrauchs, insbesondere wenn die Parallelisierung nicht effizient implementiert ist. Dies ist besonders relevant in Umgebungen, in denen Energieeffizienz ein wichtiges Anliegen ist, wie beispielsweise in mobilen Geräten oder Rechenzentren.

\section{Parallelisierung in vorwärtsgerichteten Netzwerken}
\label{sec:Grundlagen_Parallelisierung_Neuronale_Netze}
\subsection{Thread- und Prozessparallelsierung}
\label{sec:Grundlagen_Thread_Parallelisierung}
Für die parallele Ausführung des Trainings mehrerer Netzwerke unabhängig voneinander spielt die Thread- und Prozessparallelisierung eine bedeutende Rolle. Diese Techniken bieten Mechanismen, um das Training der Netzwerke auf mehrere Threads oder Prozesse aufzuteilen, was die Effizienz und Geschwindigkeit des Trainings verbessern kann \citep{Flynn_Computer_Organizations_and_their_Effectiveness}.

Thread-Parallelisierung bezieht sich auf die Aufteilung des Trainingsprozesses eines Netzwerks in mehrere Threads, die gleichzeitig auf einem einzigen Prozessorkern oder auf mehreren Kernen eines Mehrkernprozessors ausgeführt werden können. In diesem Szenario ermöglicht die Thread-Parallelisierung das gleichzeitige Training mehrerer Netzwerke, wobei jeder Thread sich auf das Training eines bestimmten Netzwerks konzentriert. Dies kann die Gesamttrainingszeit reduzieren und die Auslastung der verfügbaren Prozessorressourcen optimieren \citep{Flynn_Computer_Organizations_and_their_Effectiveness}.

Prozessparallelisierung hingegen umfasst die Aufteilung des Trainingsprozesses in mehrere unabhängige Prozesse, die auf verschiedenen Prozessorkernen oder sogar auf verschiedenen physikalischen Maschinen ausgeführt werden können. Bei der Prozessparallelisierung werden die Trainingsvorgänge mehrerer Netzwerke auf separaten Prozessen ausgeführt, was eine hochgradig parallele Verarbeitung und Skalierbarkeit über mehrere Computerknoten hinweg ermöglicht. Die Kommunikation zwischen den Prozessen kann über verschiedene Mechanismen wie Sockets, Messaging-Systeme oder gemeinsam genutzte Speicherbereiche erfolgen \citep{Flynn_Computer_Organizations_and_their_Effectiveness}.

\subsection{Implementierung von Parallelisierungstechniken}
\label{sec:Grundlagen_Parallelisierung_Implementierung}
Für die Implementierung von Thread- und Prozessparallelisierung in vorwärtsgerichteten Netzwerken können verschiedene Ansätze verfolgt werden. Eine gängige Methode besteht darin, parallele Bibliotheken oder Frameworks zu verwenden, die bereits implementierte Funktionen für die Thread- und Prozessverwaltung bereitstellen. Beispiele hierfür sind die Verwendung von OpenMP, CUDA oder MPI, je nach den Anforderungen der Anwendung und der zugrunde liegenden Hardwarearchitektur.

Bei der Implementierung von Thread-Parallelisierung können Entwickler Thread-Pools verwenden, um die Ressourcennutzung zu optimieren und die Thread-Erstellungskosten zu minimieren. Die Aufgaben werden in Threads aufgeteilt und in einem Pool von vorab erstellten Threads ausgeführt, was die Ausführungszeit der Aufgaben reduziert und die Gesamtperformance verbessert.

Für die Prozessparallelisierung ist die Implementierung von Mechanismen zur Kommunikation und Koordination zwischen den verschiedenen Prozessen entscheidend. Dies kann die Verwendung von Sockets, Messaging-Systemen wie ZeroMQ oder die gemeinsame Nutzung von Speicherbereichen umfassen, um Daten zwischen den Prozessen auszutauschen und den Trainingsfortschritt zu synchronisieren.

\subsection{Auswirkungen auf die Leistungsfähigkeit}
\label{sec:Grundlagen_Parallelisierung_Leistungsfähigkeit}
Ein wesentlicher Vorteil besteht darin, dass durch die parallele Ausführung mehrerer Netzwerke (mit verschiedenen Seeds) gleichzeitig eine Vielzahl von Trainingsdurchläufen durchgeführt werden kann. Dies ermöglicht es, eine breite Palette von Modellen zu trainieren und verschiedene hyperparameterabhängige Variationen zu erkunden, um letztendlich das optimale Modell zu identifizieren. Durch die gleichzeitige Ausführung dieser Trainingsläufe können Entwickler Zeit sparen und schneller zu aussagekräftigen Ergebnissen gelangen.

Des Weiteren bietet die parallele Ausführung die Möglichkeit, Inferenzoperationen gleichzeitig durchzuführen. Mehrere Eingaben können gleichzeitig an duplizierte Netzwerke weitergeleitet werden, um eine simultane Auswertung zu ermöglichen. Dies beschleunigt nicht nur den Inferenzprozess erheblich, sondern ermöglicht auch eine effizientere Nutzung der verfügbaren Hardwareressourcen.

Ein weiterer Vorteil besteht in der verbesserten Skalierbarkeit der Anwendung. Durch die Nutzung von Thread- oder Prozessparallelisierung kann die Anwendung problemlos auf mehreren Rechenknoten oder sogar in Cloud-Umgebungen skaliert werden. Dies ermöglicht es, die Trainings- und Inferenzkapazitäten je nach Bedarf flexibel anzupassen und die Gesamtperformance der Anwendung zu optimieren.
