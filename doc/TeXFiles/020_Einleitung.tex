\chapter{Einleitung}
\label{ch:Einleitung}

Die Informatik schöpft oft Inspiration aus der Natur, sei es durch die Nachahmung tierischer Bewegungen bei Robotern, die Organisation von Multiagentensystemen in vogelähnlichen Schwärmen oder die Anwendung evolutionärer Algorithmen zur Simulation natürlicher Prozesse. Doch eines der faszinierendsten Phänomene der Natur ist das menschliche Gehirn und seine Fähigkeit, aus Erfahrung zu lernen. Dieses komplexe Organ beschäftigt Wissenschaftler seit langem, und die Suche nach Möglichkeiten, seine Lernfähigkeit zu simulieren, hat zu bedeutenden Fortschritten geführt.

Ein zentrales Element im Gehirn ist das Neuron, das als Grundbaustein für die Informationsverarbeitung dient. Ein Neuron besteht aus einem Zellkörper, mehreren Dendriten und einem Axon. Die Dendriten empfangen elektrische Signale von vorgeschalteten Neuronen und fungieren somit als Eingangsebene für Informationen. Diese eingehenden Signale werden zum Zellkörper weitergeleitet, wo sie aufsummiert werden. Wenn ein bestimmter Schwellenwert überschritten wird, leitet das Neuron das elektrische Signal über das Axon weiter. Das Axon wiederum steht über den synaptischen Spalt mit den Dendriten nachgeschalteter Neuronen in Verbindung. Am synaptischen Spalt werden die elektrischen Signale in chemische Botenstoffe umgewandelt und ausgeschüttet. Je nach Menge der ausgeschütteten Botenstoffe entstehen in den Dendriten der nachgeschalteten Neuronen elektrische Signale, wodurch Informationen zwischen Neuronen ausgetauscht werden. Das Axon agiert somit als Ausgangsebene für Informationen.

Inspiriert von diesem biologischen Vorbild entwickelte Frank Rosenblatt 1958 das mathematische Modell des einfachen Perzeptrons - ein künstliches Neuron, das auch heute noch als Grundlage für die Entwicklung künstlicher neuronaler Netzwerke dient. Diese Netzwerke können mithilfe von Lernalgorithmen trainiert werden, um vielfältige Klassifikations- und Regressionsprobleme zu lösen. Heutzutage finden künstliche neuronale Netzwerke in verschiedenen Bereichen Anwendung, darunter die Bioinformatik, Bild- und Spracherkennung sowie die Finanzprognose , um nur einige zu nennen.

Parallel zur Weiterentwicklung von neuronalen Netzwerken ist die Frage der Skalierbarkeit und Effizienz von entscheidender Bedeutung. Insbesondere die Verarbeitung großer Datenmengen erfordert effiziente Algorithmen und Techniken zur Parallelisierung. In dieser Arbeit werden wir uns daher auch mit der Parallelisierung von neuronalen Netzen befassen und untersuchen, wie diese Techniken die Leistung und Effizienz unserer Implementierungen beeinflussen.

\section{Aufgabenstellung}
\label{sec:Einleitung_Aufgabenstellung}
In Anbetracht der weitverbreiteten Verwendung von Mehrkernprozessoren in modernen Computersystemen und der dadurch ermöglichten Parallelisierung zur Steigerung der Leistungsfähigkeit, ist es von besonderem Interesse, die Leistung der bestehenden N++ Bibliothek für maschinelles Lernen durch Parallelisierung zu verbessern. Die N++ Bibliothek ist in C++ implementiert, weshalb die Parallelisierung mithilfe von Threads realisiert werden soll. Eine zentrale Herausforderung besteht darin, geeignete Stellen im Algorithmus zu identifizieren, die von der Parallelisierung profitieren könnten. Hierbei können bereits vorhandene Bachelorarbeiten und Implementierungen als Vergleich herangezogen werden.

Die vorliegende Arbeit zielt darauf ab, die erzielte Leistungsverbesserung durch die Parallelisierung zu untersuchen und zu quantifizieren. Durch die Implementierung der Parallelisierung und die anschließende Ausführung von ausgewählten Algorithmen des maschinellen Lernens auf geeigneten Datensätzen, können wir den Effekt der Parallelisierung auf die Leistung der N++ Bibliothek evaluieren. Zudem werden wir die Auswirkungen verschiedener Parameter und Konfigurationen im Zusammenhang mit der Parallelisierung analysieren, um ein umfassendes Verständnis der Leistungsverbesserung durch Parallelisierung zu erlangen.

Insgesamt strebt diese Arbeit danach, nicht nur die technische Umsetzung der Parallelisierung zu präsentieren, sondern auch deren Auswirkungen auf die Leistungsfähigkeit der N++ Bibliothek für maschinelles Lernen zu untersuchen und zu bewerten.

\section{Gliederung}
\label{sec:Einleitung_Gliederung}


% ...



%Am Ende des Einführungskapitels ein kurzer Überblick über die kommenden Kapitel in einem Absatz (Beschreibung der Struktur der Thesis):

%...