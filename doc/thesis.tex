%%%%%%%%%%%%%%%%%%%%%%%%%%%%%%%%%%%%%%%%%%%%%%%%%%%%%%%%%%%%%%%%%%%%%%%%%%%%%%%%%%%%%%
% Template fuer Abschlussarbeiten von Studierenden der 
% Frankfurt University of Applied Sciences
% 
% erstellt von: Prof. Dr.-Ing. Thomas Hollstein
% 
% Last revision: 22.06.2023
%
%%%%%%%%%%%%%%%%%%%%%%%%%%%%%%%%%%%%%%%%%%%%%%%%%%%%%%%%%%%%%%%%%%%%%%%%%%%%%%%%%%%%%%


\documentclass[
    %twoside, 
    %openright,
  titlepage,
  numbers=noenddot,
  headinclude,
    %1headlines,
  footinclude=true,
    %cleardoublepage=empty,
    %BCOR=5mm,
  fontsize=20pt,%20pt,
  paper=a4,
    %letterpaper
    %a4paper,
  ngerman,
  american,
    %table
  ]
% Dokumententyp: (legt grundlegende Formatierrichtlinien fest)
    %{book}
    %{report}
{report}
    %{scrreprt}

%%%%%%%%%%%%%%%%%%%%%%%%%%%%%%%%%%%%%%%%%%%%%%%%%%%%%%%%%%%%%%%%%%%%%%%%%%%%%%%%%%%%%%


   

% Farbigen Text
\usepackage{xcolor}
\usepackage{adjustbox}

\parindent = 0pt  % Neue Absaetze nicht einruecken
\parskip = 1ex % Neue Absaetze: 1/2 Zeile Abstand





%%%%%%%%%% zu nutzende Pakete einbinden: %%%%%%%%%%

%%%%% Package Kommentare:



\usepackage{comment}
\usepackage[colorinlistoftodos]{todonotes}
\newcommand{\mycomment}[1]{\todo[inline,linecolor=green,backgroundcolor=yellow!25,bordercolor=green,caption={}]{todo: #1}}



%%%%% package Sprachunterstuetzung:

\usepackage{babel}[ngerman]
\usepackage{minted}
\usepackage{csquotes}

%%%%% Package Multirow ermoeglicht, dass sich ein Kaestchen in einer Tabelle ueber mehrere Zeilen erstrecken kann:

\usepackage{multirow}

% Beispiele unter: https://texblog.org/2012/12/21/multi-column-and-multi-row-cells-in-latex-tables/

%Vereinigung von Feldern in Tabellen:
%\multicolumn{number cols}{align}{text} % align: l,c,r
%\multirow{number rows}{width}{text}

%%%% Serifenloser Textstil fuer das ganze Dokument: 

\RequirePackage[sfdefault,lf]{carlito}
\usepackage{lmodern}
% \RequirePackage[T1]{fontenc}
% to imitate Calibri:
\renewcommand*\familydefault{\sfdefault} %% Base font of the document is to be sans serif

%%%%%% Grafiken einbinden
\usepackage{graphicx}
% https://golatex.de//wiki/%5cincludegraphics

%%%%%% Mathematik-Paket AMSMath:
\usepackage[fleqn,reqno]{amsmath}

\usepackage{setspace}


%%%%%% Seitengeometrie einstellen:
% https://tex.stackexchange.com/questions/344241/logo-as-header-using-fancyhdr-package
\usepackage{geometry}
\geometry{verbose,
          bmargin=2.5cm,
          lmargin=2cm,
          rmargin=2cm
          %footskip=-25pt
          }

%%%%%% die Höhe des Top-Margin errechnen:
\newlength\mytopmargin
\newsavebox{\headbox}\savebox{\headbox}{
    %\includegraphics[width=0.3\textwidth]{Figures/xxx.png} \hfill
    \raisebox{-1ex}{\includegraphics[width=0.1\textwidth]{Figures/fra-uas_logo.pdf}}
}    
\setlength{\mytopmargin}{\totalheightof{\usebox{\headbox}}+2cm}
\geometry{verbose,
          tmargin=\mytopmargin,
          headheight=1.1\mytopmargin,
          footskip=9ex
}

% Flexiblere Tabellen
\usepackage{tabularx}
\def\tabularxcolumn#1{m{#1}}

% Boxen für Lickert Skalen
\usepackage{wasysym}
\newcommand\insq[1]{%
    \Square\ #1\quad%
}

% Durchstreichen von Text
\usepackage[normalem]{ulem} % mit \sout



%%%%%% Gesamtseitenzahl verwenden:
\usepackage{totpages}

%%%%%% Kalkulationen:
%https://tex.stackexchange.com/questions/30081/how-can-i-sum-two-values-and-store-the-result-in-other-variable
\usepackage{tikz}
\usetikzlibrary{math}

%%%%%% Gestaltung von Kopf- und Fusszeilen:
% https://tex.stackexchange.com/questions/344241/logo-as-header-using-fancyhdr-package
\usepackage{fancyhdr}
\pagestyle{fancy}  % Eigener Seitenstil
\fancyhf{}         % Alle Kopf- und Fußzeilenfelder bereinigen
%\fancyhead[L]{} % Kopfzeile links
\fancyhead[l]{\leftmark
   %\makebox[0.3\textwidth]{\includegraphics[width=0.3\textwidth]{Figures/xxx.png}}
}  
\fancyhead[c]{\hspace*{0.15\textwidth}\rightmark}
%\fancyhead[C]{\usebox\headbox}                        % Zentrierte Kopfzeile
\fancyhead[R]{
  \makebox[0.2\textwidth]{\raisebox{-1ex}{\hspace*{14ex}\includegraphics[width=0.1\textwidth]{Figures/fra-uas_logo.pdf}}}
}  % Kopfzeile rechts
\renewcommand{\headrulewidth}{0.4pt} % Obere Trennlinie
%\fancyfoot[L]{\today}
\fancyfoot[C]{\ThesisTitleShort} 
%\fancyfoot[R]{Seite \thepage ~von \ref{TotPages}}  % Seitennummer
\fancyfoot[R]{\thepage} % Seitennummer 
\renewcommand{\footrulewidth}{0.4pt} % Untere Trennlinie
\setlength{\mytopmargin}{\totalheightof{\usebox\headbox} +2cm}
%Unterschied zwischen geraden/ungeraden Seiten:
%\fancyhead[OR]{} % "O" steht für "odd", also ungerade Seiten
%\fancyhead[ER]{} % "E" für "even", also gerade Seiten.




%%%%% Erweiterte Formate für Listen/Aufzaehlungen:
\usepackage{paralist}
%Default-Items fuer die vier moeglichen Verschachtelungsebenen:
\setdefaultitem{}{\textbullet}{$\star$}{}

%%%%% FRA-UAS CI Farben:
%\definecolor{airforceblue}{rgb}{0.36, 0.54, 0.66}
% CI-Farben FRA-UAS (blau):
\definecolor{FRAUAS_Blue_Dark}{RGB}{45, 137, 204}
\definecolor{FRAUAS_Blue_Light}{RGB}{182, 210, 228}
% CI-Farben FRA-UAS: FB1
\definecolor{FRAUAS_FB1_Dark}{RGB}{124, 128, 52}
\definecolor{FRAUAS_FB1_Light}{RGB}{213, 213, 179}
% CI-Farben FRA-UAS: FB2
\definecolor{FRAUAS_FB2_Dark}{RGB}{255, 158, 27}
\definecolor{FRAUAS_FB2_Light}{RGB}{251, 221, 173}
% CI-Farben FRA-UAS: FB3
\definecolor{FRAUAS_FB3_Dark}{RGB}{196, 213, 42}
\definecolor{FRAUAS_FB3_Light}{RGB}{237, 240, 166}
% CI-Farben FRA-UAS: FB4
\definecolor{FRAUAS_FB4_Dark}{RGB}{204, 31, 47}
\definecolor{FRAUAS_FB4_Light}{RGB}{240, 166, 183}

%%%%% Sektionstitel nach CI-Farben einfärben:
\usepackage{titlesec}
\titleformat{\section}
{\color{FRAUAS_Blue_Dark}\normalfont\Large\bfseries} %Titel
{\color{FRAUAS_Blue_Dark}\thesection}{1em}{}
\titleformat{\subsection}
{\color{FRAUAS_Blue_Dark}\normalfont\large\bfseries} %Titel
{\color{FRAUAS_Blue_Dark}\thesubsection}{1em}{}

%\usepackage{appendix}

%%%%% Erweiterte Bibliographie-Stile:
%\usepackage{harvard}
% Title Page
% (wir generieren den Titel per Handlayout und verwenden
% daher die folgenden Befehle nicht)
%\title{}
%\author{}

% https://www.overleaf.com/learn/latex/Hyperlinks
\usepackage{hyperref}
\hypersetup{
    %hyperindex=true,
    %linktocpage=true, % Seitenzahl statt Titel verlinkt
    colorlinks=true,
    linkcolor=black,
    filecolor=black,      
    urlcolor=black,
    pdftitle={Thesis},
    pdfpagemode=FullScreen,
    citecolor=black,
    }



%\usepackage[hyphens]{url}  %% stellt \url{} zur Verfuegung

%%%%% modernes BibLaTeX mit biber %%%%%
% https://golatex.de/viewtopic.php?t=13917
%\usepackage[style=ieee-alphabetic,
%backend=biber, natbib=true]{biblatex}

% Literatur nach Erscheinen im Text sortiert
%\usepackage[sorting=none, style=numeric,backend=biber, natbib=true]{biblatex}
%\usepackage[style=apa, backend=biber, natbib=true, sorting=nyt, sortcites=false]{biblatex}
\usepackage[style=numeric, backend=biber, natbib=true, sorting=none, sortcites=false]{biblatex}
% Ohne eingestellte Sortierung
%\usepackage[ style=numeric,
%backend=biber, natbib=true]{biblatex}


\addbibresource{bibliography.bib}

% Immer shortautor anzeigen, falls vorhanden
\makeatletter
\def\cbx@apa@ifnamesaved{\@firstoftwo}
\makeatother

% Zeilenabstand für das ganze Dokument (1.0 = Normalwert):
\renewcommand{\baselinestretch}{1.0}

\usepackage{fontsize}
  \changefontsize[14]{14}

\usepackage[acronym]{glossaries}
\makeglossaries

\usepackage{siunitx}

%%%%%%%%%%%%%%%%%%%%%%%%%%%%%%%%%%%%%%%%%%%%
%%%%% hier beginnt das eigentliche Dokument 
%%%%%%%%%%%%%%%%%%%%%%%%%%%%%%%%%%%%%%%%%%%%


\begin{document}

%%%%% Settings einbinden:
\newcommand{\myName}{Luca Andrea John Vinciguerra}
\newcommand{\ThesisTitle}{Nebenläufige Algorithmen im maschinellen Lernen: Analyse,
Implementierung und vergleichende Untersuchungen zur Parallelisierung
einer Bibliothek für künstliche neuronale Netze}
\newcommand{\ThesisTitleShort}{Nebenläufige Algorithmen im maschinellen Lernen}
%\newcommand{\ThesisSubtitle}{This is the subtitle of the thesis}
\newcommand{\ThesisDegree}{Bachelor of Science (B.Sc.)} 
%\newcommand{\ThesisDegree}{Bachelor of Arts (B.A.)}
%\newcommand{\ThesisDegree}{Master of Science (M.Sc.)}
%\newcommand{\ThesisDegree}{Master of Arts (M.A.)}
\newcommand{\myStudentId}{1296334}
\newcommand{\Supervisor}{Prof. Dr. Thomas Gabel}
\newcommand{\CoSupervisor}{Prof. Dr. Christian Baun}
\newcommand{\Faculty}{Fachbereich 2: Informatik und Ingenieurwissenschaften}
\newcommand{\University}{Frankfurt University of Applied Sciences}
\newcommand{\UniversityLocation}{Frankfurt}
\newcommand{\ThesisDeliveryDate}{21. Mai 2024}
\newcommand{\CompanyName}{NoCompanyInvolved}  % don't modify this line, if no company is involved


%%%%%%%%%%%%%%%%%%%%% FOR ENGLISH LANGUAGE THESIS: %%%%%%%%%%%%%%%%%%%%

% By default the thesis language is german
% if you want to set it to ENGLISH, then UNCOMMENT the FOLLOWING LINE by removing the leading "%":

%\newcommand*{\ThesisLanguageIsEnglish}{}


\sloppy %Formatierungsueberstaende am Zeilenende vermeiden
% https://latexref.xyz/_005cfussy-_0026-_005csloppy.html

\frenchspacing  
%Ein Leerzeichen nach Satzende
%https://texwelt.de/fragen/1154/was-ist-french-spacing-was-macht-frenchspacing

%\raggedbottom   
%Standard is \flushbottom, dass heisst
%alle Seiten werden so gedehnt, dass sie 
%gleich hoch sind 
%Schaltet man \raggedbottom ein, ist dies
%nicht so

\ifdefined\ThesisLanguageIsEnglish
\selectlanguage{american}
\else
\selectlanguage{ngerman} % ngerman, american
\fi
%Deutsch nach neuer Rechtschreibung als
%Standardsprache fuer das Dokument einstellen

%\renewcommand*{\bibname}{new name}
%\setbibpreamble{}

% Numerierungstiefe setzen:
% 2: bis subsection (Standard)
% 3: bis subsubsection
\setcounter{secnumdepth}{3} %setzt die Numerierungstiefe



%\renewcommand{\thepage}{\Roman{page}}
\pagestyle{plain} % Seite ohne Kopf- und Fusszeilen darstellen



% hier wird die Titelseite eingebunden:
\thispagestyle{empty}
%\pdfbookmark[0]{Titelblatt}{title}

%*******************************************************
% Titlepage
%*******************************************************
%%%
%%% title page (german)
%%%
\thispagestyle{empty}
\pdfbookmark[0]{Titelblatt}{title}
\begin{titlepage}


  \vspace*{-3,5cm}
  \begin{center}
    \includegraphics[width=7.7cm]{Figures/fra-uas_logo} \\ 
  \end{center}

  \begin{center}
    \vspace{0.1cm}
    \LARGE \textbf{Frankfurt University\\ of Applied Sciences}\\
    \vspace{0.4cm}
    \Large -- \Faculty --
  \end{center}

  \vfill

  \begin{center}
    \huge \textbf{\ThesisTitle}   %%%%% >>>>> Bitte in TeXFiles/000_Settings eintragen 
  \end{center} 

  \vfill

  \ifdefined\ThesisLanguageIsEnglish 
  \begin{center}
    \Large Thesis submitted in order to obtain the academic degree\\
    \vspace{0.3cm}
    \Large \ThesisDegree   %%%%% >>>>> Bitte in TeXFiles/000_Settings eintragen 
  \end{center}
  \else
  \begin{center}
    \Large Abschlussarbeit zur Erlangung des akademischen Grades\\
    \vspace{0.3cm}
    \Large \ThesisDegree   %%%%% >>>>> Bitte in TeXFiles/000_Settings eintragen 
  \end{center}
  \fi

  \vfill

  \ifdefined\ThesisLanguageIsEnglish 
  \begin{center}
    \Large submitted on \ThesisDeliveryDate\ on\\   %%%%% >>>>> Bitte in TeXFiles/000_Settings eintragen
    \vspace{0.3cm}
    \Large \textbf{\myName}\\
    \vspace{0.3cm}
    \normalsize Student ID: \myStudentId   %%%%% >>>>> Bitte in TeXFiles/000_Settings eintragen
  \end{center}
  \else
  \begin{center}
    \Large vorgelegt am \ThesisDeliveryDate\ von\\   %%%%% >>>>> Bitte in TeXFiles/000_Settings eintragen
    \vspace{0.3cm}
    \Large \textbf{\myName}\\
    \vspace{0.3cm}
    \normalsize Matrikelnummer: \myStudentId   %%%%% >>>>> Bitte in TeXFiles/000_Settings eintragen
  \end{center}
  \fi

  \vfill

  \ifdefined\ThesisLanguageIsEnglish 
  \begin{center}
    \begin{tabular}{lll}
      First Supervisor    & : & \Supervisor \\     %%%%% >>>>> Bitte in TeXFiles/000_Settings eintragen
      Second Supervisor & : & \CoSupervisor\\    %%%%% >>>>> Bitte in TeXFiles/000_Settings eintragen
    \end{tabular}
  \end{center} 
  \else
  \begin{center}
    \begin{tabular}{lll}
      Referent    & : & \Supervisor \\     %%%%% >>>>> Bitte in TeXFiles/000_Settings eintragen
      Korreferent & : & \CoSupervisor\\    %%%%% >>>>> Bitte in TeXFiles/000_Settings eintragen
    \end{tabular}
  \end{center} 
  \fi

\newpage


  

\end{titlepage}

\input{TeXFiles/002_NDNotice}
\input{TeXFiles/003_Declaration}
\newpage

% da wir den Titel haendisch erstellt haben entfaellt der folgende Befehl:
%\maketitle

%\clearpage

%\pagestyle{headings} 
\pagestyle{fancy}   % Seite mit Kopf- und Fusszeilen darstellen
\pagenumbering{roman}  % Umschalten auf Seitenzahlen in römischer Darstellung
%\fancyfoot[R]{i}  % Seitennummer
%roemisch 1, hier per Hand eingetragen, weil es anders nicht funktioniert hat - hat sich erledigt


\tableofcontents
\clearpage
\pagenumbering{arabic}  % Umschalten auf Seitenzahlen in arabischer Darstellung
%%%%% Einbinden der Dateien für die einzelnen Sektionen:
% Verwendet man hier \include statt \input, beginnen
% neue Sektionen immer auf einer neuen Seite
%\input{TeXFiles/Abstract}

\fancyfoot[R]{\thepage}  % Seitennummer

%\myName
\setstretch{1.5}
\ifdefined\ThesisLanguageIsEnglish
\chapter*{Abstract}
\else
\chapter*{Kurzfassung}
\fi
\label{ch:Abstract}

\mycomment{Die Kurzfassung muss noch geschrieben werden}
\chapter{Einleitung}
\label{ch:Einleitung}

Die Informatik schöpft oft Inspiration aus der Natur, sei es durch die Nachahmung tierischer Bewegungen bei Robotern, der Organisation von Multiagentensystemen in vogelähnlichen Schwärmen oder der Anwendung evolutionärer Algorithmen zur Simulation natürlicher Prozesse. Doch eines der faszinierendsten Phänomene der Natur ist das menschliche Gehirn und seine Fähigkeit, aus Erfahrung zu lernen. Dieses komplexe Organ beschäftigt Wissenschaftler seit langem, und die Suche nach Möglichkeiten, seine Lernfähigkeit zu simulieren, hat zu bedeutenden Fortschritten geführt.

Ein zentrales Element im Gehirn ist das Neuron, auch Nervenzelle genannt, das als Grundbaustein für die Informationsverarbeitung dient. Das Menschliche Gehirn besitzt circa 10 Milliarden Neuronen. Jedes dieser Neuronen besteht aus einem Zellkörper, mehreren Dendriten und einem Axon. Die Dendriten empfangen elektrische Signale von davor geschalteten Neuronen und fungieren somit als Eingangsebene für Informationen. Diese eingehenden Signale werden zum Zellkörper weitergeleitet, wo sie aufsummiert werden. Wird ein bestimmter Schwellenwert überschritten, leitet das Neuron das elektrische Potenziale über das Axon weiter, welches es elektrochemisch an nachgeschaltete Neuronen über deren Dendriten weiterleitet. Das Axon agiert somit als Ausgangsebene für Informationen des Neurons. 
Durch diese Kettenreaktion können im Gehirn somit komplexe Sachverhalte verarbeitet werden \citep{Praktische_Einfuhrung_in_neuronale_Netze}.

Inspiriert von diesem biologischen Vorbild entwickelte Frank Rosenblatt 1958 das Modell des Perzeptrons - einem künstlichen Neuron, welches die Grundlage für die Entwicklung heutiger künstlicher neuronaler Netzwerke darstellt \citep{Rosenblatt_Perceptron}. Diese Netzwerke können mithilfe von Lernalgorithmen trainiert werden, um vielfältige Probleme zu bewältigen, welche mit konventionellen Computeralgorithmen wenn überhaupt nur schwer zu lösen sind. Mittlerweile tragen künstliche neuronale Netzwerke mitunter in verschiedensten Branchen zu der Realisierung von Software in vielfältigen Anwendungsgebieten bei.

Aufgrund der großen Menge an Daten und benötigten Rechenleistung für die Darstellung von neuronalen Netzwerken in Computern ist die Frage der Skalierbarkeit und Effizienz der neuronalen Netzwerke von entscheidender Bedeutung. Insbesondere die Verarbeitung großer Datenmengen erfordert effiziente Algorithmen und Techniken zur Parallelisierung, um den gegebenen Anforderungen beispielsweise im Bezug auf Latenz gerecht zu werden. In dieser Arbeit wird daher die Parallelisierung von neuronalen Netzen thematisiert und untersucht, wie diese Techniken die Leistung und Effizienz beeinflussen.

\section{Aufgabenstellung}
\label{sec:Einleitung_Aufgabenstellung}
In Anbetracht der stagnierenden Entwicklung der Taktrate aufgrund des Annäherns an das physikalische Limit konnten in den letzten Jahren keine großen Verbesserungen in der Einkernleistung erzielt werden. Deshalb setzten Prozessorhersteller weit verbreitet auf Mehrkernprozessoren, um Leistungssteigerungen zu ermöglichen \citep{Chip_makers_turn_to_multicore}. Angesichts der möglichen Leistungssteigerung durch effizientes Nutzen aller verfügbaren Kerne ist es von besonderem Interesse, die Leistung der bestehenden n++-Bibliothek für maschinelles Lernen durch Parallelisierung zu verbessern. Die n++-Bibliothek ist in C++ implementiert, weshalb die Parallelisierung mithilfe von Threads realisiert werden soll. Eine zentrale Herausforderung besteht darin, geeignete Stellen in der Bibliothek als auch in Anwendungsprogrammen zu identifizieren, die von der Parallelisierung profitieren könnten. Hierbei werden vorhandene Vorarbeiten \citep{thesis_Artur_Brening} und Implementierungen als Vergleich herangezogen und gegebenenfalls optimiert.

Diese Arbeit zielt darauf ab, die potenziell erzielten Leistungsverbesserungen durch die Parallelisierung zu untersuchen und zu quantifizieren. Durch die Implementierung der Parallelisierung und die anschließende Ausführung von Algorithmen der Vorarbeit kann der Effekt der Parallelisierung auf die Leistung von Programmen, welche die n++-Bibliothek verwenden, evaluiert werden. Zudem werden die Auswirkungen verschiedener Parameter und Konfigurationen im Zusammenhang mit der Parallelisierung analysieren, um ein umfassendes Verständnis der Leistungsverbesserung durch Parallelisierung zu erlangen.

Insgesamt strebt diese Arbeit danach, nicht nur die technische Umsetzung der Parallelisierung zu präsentieren, sondern auch deren Auswirkungen auf die Leistungsfähigkeit der n++-Bibliothek für maschinelles Lernen zu untersuchen und zu bewerten.

\section{Gliederung}
\label{sec:Einleitung_Gliederung}


% ...



%Am Ende des Einführungskapitels ein kurzer Überblick über die kommenden Kapitel in einem Absatz (Beschreibung der Struktur der Thesis):

%...
\chapter{Grundlagen}
\label{ch:Grundlagen}

Dieses Kapitel legt die Grundlagen für künstliche neuronale Netze dar, indem es detailliert auf ihre Struktur und Funktionsweise eingeht, mit Fokus auf vorwärtsgerichtete Netze. Dabei wird ein umfassender Überblick über die potenziellen Anwendungsbereiche von künstlichen neuronalen Netzen gegeben.

Des Weiteren wird die Thematik der Parallelisierung sowohl im allgemeinen Kontext als auch speziell im Zusammenhang mit neuronalen Netzen erläutert. Es werden die Vor- und Nachteile dieser Technik beleuchtet und die gängigsten Methoden zur Parallelisierung von Berechnungen in neuronalen Netzen werden ausführlich diskutiert.
Die Nutzung von Grafikprozessoren (GPUs) für parallele Berechnungen wird dabei angesprochen, jedoch liegt der Fokus auf den grundlegenden Prinzipien der Parallelisierung von neuronalen Netzen und deren Implementierung mittels Thread- und Prozessparallelsierung.

\section{Grundprinzipien von neuronalen Netzen}
\label{sec:Grundlagen_Grundprinzipien_neuronale_Netze}
Neuronale Netze sind ein wesentlicher Bestandteil des maschinellen Lernens und der künstlichen Intelligenz. Sie sind inspiriert von der Funktionsweise des menschlichen Gehirns und bestehen aus einer Ansammlung miteinander verbundener Knoten, die genau wie beim menschlichen Gehirn als Neuronen bezeichnet werden \citep{Manuela_Kunstliche_Intelligenz}.

Ein mehrlagiges neuronales Netz ist in verschiedene Schichten organisiert, die jeweils spezifische Funktionen erfüllen. Die erste Schicht wird oft als Eingangsschicht bezeichnet und empfängt die Rohdaten oder Merkmale, die dem Netz präsentiert werden. Diese Daten werden dann durch das Netz weitergeleitet, wobei jede Schicht eine spezifische Transformation durchführt.
Zwischen der Eingangsschicht und der Ausgangsschicht können mehrere versteckte Schichten vorhanden sein. Diese versteckten Schichten sind entscheidend für die Fähigkeit des Netzes, komplexe Muster zu lernen und abstrakte Merkmale zu extrahieren. Jede Schicht lernt auf unterschiedlichen Abstraktionsebenen und trägt zur schrittweisen Verbesserung der Leistung des Netzes bei.

Die Funktionsweise eines neuronalen Netzes lässt sich allgemein in zwei Hauptphasen unterteilen: Vorwärtspropagierung und Rückwärtspropagierung. Während der Vorwärtspropagierung fließen die Eingabedaten vorwärts durch das Netz, beginnend mit den Neuronen der Eingabeschicht, welche die Rohdaten empfangen, und endend mit den Neuronen der Ausgabeschicht, welche die Vorhersagen oder Klassifikationen des Netzes abbilden. Jedes Neuron ist mit anderen Neuronen verbunden, und diese Verbindungen sind mit Gewichten versehen, die die Stärke der Verbindung zwischen den Neuronen darstellen  \citep{Manuela_Kunstliche_Intelligenz}.

Während der Vorwärtspropagierung durchläuft jede Eingabe eine Reihe von Schichten im Netz, wobei jede Schicht aus einer bestimmten Anzahl von Neuronen besteht. Jedes Neuron einer Schicht erhält Eingaben von den Neuronen der vorherigen Schicht, multipliziert diese Eingaben mit den entsprechenden Gewichten und summiert sie, wie in Gleichung \ref{eq:perceptron_weighted} gezeigt. Anschließend wird eine Aktivierungsfunktion auf die gewichtete Summe angewendet, um die Ausgabe des Neurons zu berechnen, die dann an die Neuronen der nächsten Schicht weitergeleitet wird.

\begin{equation}
    o=f\left(\sum_{k=1}^{n}i_{k}\cdot W_{k}\right)
    \label{eq:perceptron_weighted}
\end{equation}
\todo{text}

Die Ausgangsschicht liefert schließlich die Ergebnisse der Netzberechnungen, sei es in Form einer Klassifikation, Regression oder einer anderen Art der Informationsverarbeitung, je nach den Anforderungen der spezifischen Anwendung. Durch die Strukturierung des Netzes in Schichten und die Festlegung spezifischer Funktionen für jede Schicht kann das neuronale Netz effizient Informationen verarbeiten und komplexe Probleme lösen.

Die Rückwärtspropagierung ist der Prozess, bei dem das Netz lernt, indem es seine Gewichte entsprechend der Fehler zwischen den tatsächlichen und den vorhergesagten Ausgaben anpasst. Dies geschieht durch die Berechnung von Gradienten mit Hilfe des Backpropagation-Algorithmus \todo{add citation} und die Anpassung der Gewichte mithilfe eines Optimierungsalgorithmus wie dem Gradientenabstiegsverfahren.

\section{Vorwärtsgerichtete und rückwärtsgerichtete Netze}
\label{sec:Grundlagen_vorwarts_Netze}
Neuronale Netze können in zwei Hauptkategorien unterteilt werden: vorwärtsgerichtete (feedforward) und rückgekoppelte (feedback) Netze.
Vorwärtsgerichtete Netzwerke sind die am häufigsten verwendete Architektur in der neuronalen Netzwerkwissenschaft. In diesen Netzwerken fließen die Informationen nur in eine Richtung, von der Eingangsschicht zur Ausgangsschicht, ohne Rückkopplungsschleifen. Das bedeutet, dass die Ausgabe jedes Neurons in einer Schicht nur von den Eingaben der vorhergehenden Schicht abhängt und nicht von den Ausgaben der Neuronen derselben Schicht oder einer späteren Schicht.

Dieses einfache Flussmuster ermöglicht eine effiziente Berechnung und einfache Interpretation der Ergebnisse. Vorwärtsgerichtete Netzwerke eignen sich besonders gut für Anwendungen wie Klassifikation und Regression, bei denen eine direkte Zuordnung von Eingaben zu Ausgaben erfolgt.

Im Gegensatz dazu haben rückgekoppelte Netzwerke Rückkopplungsschleifen, die es ermöglichen, Informationen sowohl vorwärts als auch rückwärts durch das Netzwerk zu propagieren. Diese Art von Netzwerken, auch als rekurrente neuronale Netzwerke (RNNs) bekannt, sind besonders gut geeignet für die Verarbeitung sequenzieller Daten, bei denen der Kontext und die zeitliche Abfolge der Eingaben wichtig sind.

RNNs sind in der Lage, vergangene Informationen zu berücksichtigen und sie in die aktuelle Berechnung einzubeziehen, was sie besonders nützlich für Aufgaben wie Sprachverarbeitung, Zeitreihenanalyse und maschinelles Übersetzen macht.

Die Wahl zwischen vorwärtsgerichteten und rückwärtsgerichteten Netzwerken hängt von den spezifischen Anforderungen der Anwendung ab. Während vorwärtsgerichtete Netzwerke gut für statische Daten und klare Ein-Aus-Beziehungen geeignet sind, bieten rückwärtsgerichtete Netzwerke eine größere Flexibilität und sind besser für die Verarbeitung dynamischer Daten geeignet.

\section{Anwendungsfälle von neuronalen Netzen}
\sectionmark{Anwendungsfälle neuronaler Netze}
\label{sec:Grundlagen_Anwendungsfälle}

Neuronale Netzwerke haben sich besonders im letzten Jahrzehnt als äußerst vielseitige Instrumente erwiesen und finden breite Anwendung in diversen Branchen. Prominente Einsatzgebiete umfassen:

\begin{itemize} 

\item Bilderkennung und Computer Vision: Eine der bekanntesten Anwendungen von neuronalen Netzwerken ist die Bilderkennung, mit der Objekte, Gesichter, Muster und weitere Elemente in Bildmaterial identifiziert und klassifiziert werden können \citep{LeCun_Deep_Learning}. Als Beispiel sind zum Beispiel die simple Gesichtserkennung zum Entsperren von Smartphones, sowie die Erkennung von Personen auf Bildern in verschiedenen Cloudservices zu nennen.

\item Natürliche Sprachverarbeitung: In der Natürlichen Sprachverarbeitung, auch NLP genannt, kommen neuronale Netzwerke zum Einsatz, um menschenähnliche Sprache zu verstehen, zu interpretieren und zu generieren. Anwendungen erstrecken sich von Chatbots und virtuellen Assistenten bis hin zu maschineller Übersetzung und Analyse von Inhalt in sozialen Medien.

\item Mustererkennung und Prognose: Neuronale Netzwerke finden in diversen Domänen Anwendung bei der Mustererkennung und Prognose. Dies umfasst unter anderem das Finanzwesen, Gesundheitswesen und den Verkehr. Sie ermöglichen die Identifikation von Mustern in umfangreichen Datensätzen und die Vorhersage zukünftiger Ereignisse. Weitere wichtige Anwendungsgebiete sind die Prognose von Wetterdaten und die Analyse von Verspätungsdaten öffentlicher Verkehrsmittel.

\item Autonome Fahrzeuge: In der Automobilindustrie spielen neuronale Netzwerke eine Schlüsselrolle bei der Entwicklung autonomer Fahrzeuge. Sie werden eingesetzt, um Hindernisse zu erkennen, Verkehrssituationen zu verstehen, Routen zu planen und Fahrzeugfunktionen zu steuern.

\item Medizinische Diagnose: Neuronale Netzwerke werden in der medizinischen Bildgebung verwendet, um Krankheiten wie Krebs auf Röntgen- und MRT-Scans zu identifizieren. Sie unterstützen Ärzte auch bei der Diagnose von Krankheiten und der Vorhersage von Behandlungsergebnissen anhand von Patientendaten \citep{LeCun_Deep_Learning}.

\item Finanzwesen: Im Finanzsektor kommen neuronale Netzwerke für die Kreditrisikobewertung, Betrugserkennung, Handelsstrategien und Marktanalyse zum Einsatz. Sie unterstützen Finanzinstitute bei fundierten Entscheidungen und der Minimierung von Risiken.

\end{itemize}

Diese Anwendungsbereiche verdeutlichen die Vielseitigkeit und transformative Kraft neuronaler Netzwerke in diversen Branchen und Disziplinen. Durch kontinuierliche Forschung und Entwicklung werden ihre Fähigkeiten kontinuierlich erweitert, was zu neuen Anwendungen führt.

\todo{eventuell kürzen oder weglassen}

\section{Einführung in die Parallelisierung}
\label{sec:Grundlagen_Parallelisierung}
Die Parallellisierung stellt einen zentralen Ansatz dar, um die Leistungsfähigkeit von Computersystemen zu steigern, insbesondere angesichts der Tatsache, dass moderne CPUs und GPUs über eine wachsende Anzahl von Kernen verfügen. Kern der Parallellisierung ist die simultane und unabhängige Ausführung von Aufgaben oder Berechnungen, anstatt einer sequenziellen Abfolge. Dieser Ansatz findet breite Anwendung in verschiedenen Bereichen wie High-Performance-Computing (HPC), Datenverarbeitung, Simulationen, künstlicher Intelligenz und weiteren \citep{Flynn_Computer_Organizations_and_their_Effectiveness}.

Eine Vielzahl von Herangehensweisen zur Parallellisierung existiert, die abhängig von der Problemstellung und der verfügbaren Hardware eingesetzt werden können. Die Task-Parallellisierung zielt darauf ab, Aufgaben auf mehrere Prozessoren oder Kerne zu verteilen. Insbesondere für Anwendungen mit vielen simultan auszuführenden Aufgaben wie parallele Suchalgorithmen oder Simulationen von physikalischen Systemen eignet sich diese Art der Parallellisierung besonders \citep{Flynn_Computer_Organizations_and_their_Effectiveness}.

Ein weiterer Ansatz ist die Datenparallellisierung, bei der ein Problem in kleinere Teile zerlegt wird, die jeweils auf unterschiedlichen Datensätzen operieren. Dieser Ansatz ist besonders effektiv für Anwendungen, die eine simultane Verarbeitung großer Datenmengen erfordern, wie beispielsweise Bildverarbeitung oder maschinelles Lernen \citep{Flynn_Computer_Organizations_and_their_Effectiveness}.

Es ist jedoch zu betonen, dass nicht alle Probleme gleichermaßen für eine Parallellisierung geeignet sind. Manche Probleme beinhalten intrinsische Abhängigkeiten oder Sequenzialität, die eine effektive Parallellisierung erschweren oder gar unmöglich machen.

\subsection{Vor- und Nachteile von Parallelisierung}
\label{sec:Grundlagen_Parallelisierung_Vorteile_Nachteile}
Die Parallelisierung bietet eine Vielzahl von Vorteilen, die zur Leistungssteigerung von Computersystemen beitragen. Einer der offensichtlichsten Vorteile ist die Verbesserung der Ausführungsgeschwindigkeit von Programmen und Berechnungen durch die gleichzeitige Ausführung von Aufgaben oder die Verarbeitung von Daten auf mehreren Prozessoren oder Kernen. Diese beschleunigte Ausführung ist insbesondere bei rechenintensiven Anwendungen von Vorteil.

Ein weiterer Vorteil der Parallelisierung liegt in ihrer Skalierbarkeit, da Aufgaben oder Daten auf mehrere Ressourcen aufgeteilt werden können, um Systeme leichter an wachsende Anforderungen anzupassen. Dies ermöglicht es, die Leistungsfähigkeit von Systemen flexibel zu erweitern, ohne dass eine komplette Neuentwicklung erforderlich ist.

Des Weiteren kann die Parallelisierung die Auslastung von Ressourcen optimieren und Engpässe reduzieren, indem sie Prozessoren oder andere Hardware-Ressourcen effizient nutzt. Dies trägt dazu bei, die Gesamtleistung des Systems zu verbessern. Eine effiziente Nutzung der verfügbaren Hardware ist nicht nur im Bezug der Systemleistung vorteilhaft zu bewerten, sondern ermöglicht es auch mehr Arbeit auf wenigeren Systemen auszuführen, da das volle Leistungspotenzial aller Systeme ausgenutzt wird. So werden auch Kosten gesenkt.

Trotz dieser Vorteile gibt es auch einige Nachteile und Herausforderungen bei der Implementierung von Parallelisierung. Ein wichtiger Aspekt sind die erhöhten Anforderungen an die Programmierung und das Systemdesign, da die Entwicklung paralleler Algorithmen und die Verwaltung paralleler Prozesse spezifisches Fachwissen erfordern. Darüber hinaus können Probleme wie Datenabhängigkeiten, Wettlaufsituationen und Synchronisationskonflikte auftreten, die die Entwicklung und Fehlerbehebung erschweren. Parallele Implementierungen sind fast ausschließlich komplexer als ihre sequenziellen Pendants. Es gibt einige Ansätze, die Parallelisierung dem Programmierer gegenüber transparent zu gestalten \citep{Sidorenko_Subway_Train_Scheduling}, jedoch stellten sich diese Bemühungen größtenteils als erfolglos heraus.

Ein weiterer Nachteil ist die potenzielle Zunahme des Energieverbrauchs, insbesondere wenn die Parallelisierung nicht effizient implementiert ist. Dies ist besonders relevant in Umgebungen, in denen Energieeffizienz ein wichtiges Anliegen ist, wie beispielsweise in mobilen Geräten oder Rechenzentren.

\section{Parallelisierung in vorwärtsgerichteten Netzwerken}
\label{sec:Grundlagen_Parallelisierung_Neuronale_Netze}
\subsection{Thread- und Prozessparallelsierung}
\label{sec:Grundlagen_Thread_Parallelisierung}
Für die parallele Ausführung des Trainings mehrerer Netzwerke unabhängig voneinander spielt die Thread- und Prozessparallelisierung eine bedeutende Rolle. Diese Techniken bieten Mechanismen, um das Training der Netzwerke auf mehrere Threads oder Prozesse aufzuteilen, was die Effizienz und Geschwindigkeit des Trainings verbessern kann \citep{Flynn_Computer_Organizations_and_their_Effectiveness}.

Thread-Parallelisierung bezieht sich auf die Aufteilung des Trainingsprozesses eines Netzwerks in mehrere Threads, die gleichzeitig auf einem einzigen Prozessorkern oder auf mehreren Kernen eines Mehrkernprozessors ausgeführt werden können. In diesem Szenario ermöglicht die Thread-Parallelisierung das gleichzeitige Training mehrerer Netzwerke, wobei jeder Thread sich auf das Training eines bestimmten Netzwerks konzentriert. Dies kann die Gesamttrainingszeit reduzieren und die Auslastung der verfügbaren Prozessorressourcen optimieren \citep{Flynn_Computer_Organizations_and_their_Effectiveness}.

Prozessparallelisierung hingegen umfasst die Aufteilung des Trainingsprozesses in mehrere unabhängige Prozesse, die auf verschiedenen Prozessorkernen oder sogar auf verschiedenen physikalischen Maschinen ausgeführt werden können. Bei der Prozessparallelisierung werden die Trainingsvorgänge mehrerer Netzwerke auf separaten Prozessen ausgeführt, was eine hochgradig parallele Verarbeitung und Skalierbarkeit über mehrere Computerknoten hinweg ermöglicht. Die Kommunikation zwischen den Prozessen kann über verschiedene Mechanismen wie Sockets, Messaging-Systeme oder gemeinsam genutzte Speicherbereiche erfolgen \citep{Flynn_Computer_Organizations_and_their_Effectiveness}.

\subsection{Implementierung von Parallelisierungstechniken}
\label{sec:Grundlagen_Parallelisierung_Implementierung}
Für die Implementierung von Thread- und Prozessparallelisierung in vorwärtsgerichteten Netzwerken können verschiedene Ansätze verfolgt werden. Eine gängige Methode besteht darin, parallele Bibliotheken oder Frameworks zu verwenden, die bereits implementierte Funktionen für die Thread- und Prozessverwaltung bereitstellen. Beispiele hierfür sind die Verwendung von OpenMP, CUDA oder MPI, je nach den Anforderungen der Anwendung und der zugrunde liegenden Hardwarearchitektur.

Bei der Implementierung von Thread-Parallelisierung können Entwickler Thread-Pools verwenden, um die Ressourcennutzung zu optimieren und die Thread-Erstellungskosten zu minimieren. Die Aufgaben werden in Threads aufgeteilt und in einem Pool von vorab erstellten Threads ausgeführt, was die Ausführungszeit der Aufgaben reduziert und die Gesamtperformance verbessert.

Für die Prozessparallelisierung ist die Implementierung von Mechanismen zur Kommunikation und Koordination zwischen den verschiedenen Prozessen entscheidend. Dies kann die Verwendung von Sockets, Messaging-Systemen wie ZeroMQ oder die gemeinsame Nutzung von Speicherbereichen umfassen, um Daten zwischen den Prozessen auszutauschen und den Trainingsfortschritt zu synchronisieren.

\subsection{Auswirkungen auf die Leistungsfähigkeit}
\label{sec:Grundlagen_Parallelisierung_Leistungsfähigkeit}
Ein wesentlicher Vorteil besteht darin, dass durch die parallele Ausführung mehrerer Netzwerke (mit verschiedenen Seeds) gleichzeitig eine Vielzahl von Trainingsdurchläufen durchgeführt werden kann. Dies ermöglicht es, eine breite Palette von Modellen zu trainieren und verschiedene hyperparameterabhängige Variationen zu erkunden, um letztendlich das optimale Modell zu identifizieren. Durch die gleichzeitige Ausführung dieser Trainingsläufe können Entwickler Zeit sparen und schneller zu aussagekräftigen Ergebnissen gelangen.

Des Weiteren bietet die parallele Ausführung die Möglichkeit, Inferenzoperationen gleichzeitig durchzuführen. Mehrere Eingaben können gleichzeitig an duplizierte Netzwerke weitergeleitet werden, um eine simultane Auswertung zu ermöglichen. Dies beschleunigt nicht nur den Inferenzprozess erheblich, sondern ermöglicht auch eine effizientere Nutzung der verfügbaren Hardwareressourcen.

Ein weiterer Vorteil besteht in der verbesserten Skalierbarkeit der Anwendung. Durch die Nutzung von Thread- oder Prozessparallelisierung kann die Anwendung problemlos auf mehreren Rechenknoten oder sogar in Cloud-Umgebungen skaliert werden. Dies ermöglicht es, die Trainings- und Inferenzkapazitäten je nach Bedarf flexibel anzupassen und die Gesamtperformance der Anwendung zu optimieren.

\chapter{Implementierung der Parallelisierung in N++}
\label{ch:Implementierung_Parallelisierung_Npp}
\chaptermark{Implementierung}

\section{Erläuterung des N++ Simulatorkerns}
\label{sec:Erlauterung_Npp}
\sectionmark{Erläuterung N++}
N++ ist ein Simulator für neuronale Netze, der als Forschungsprojekt an der Frankfurt University of Applied Sciences entwickelt wurde. Die Software ermöglicht die Simulation mehrerer neuronaler Netze und strebt danach, dem Anwender eine einfache Erweiterung der Grundfunktionen sowie eine benutzerfreundliche Schnittstelle für Anwendungsprogramme bereitzustellen.

Die Bibliothek N++ ist in C++ verfasst. Da sie seit über 20 Jahren besteht, verwendet sie größtenteils keine modernen C++-Features, unter anderem auch um die Kompatibilität mit C beizubehalten, da der Kern des Simulators auf dieser älteren Sprachversion aufbaut. Oft werden im Quellcode Funktionen der Standartbibliothek von C denen von C++ vorgezogen.

N++ ermöglicht es dem Benutzer, die Topologie des neuronalen Netzes zu spezifizieren und es an die spezifischen Anforderungen anzupassen. Hierbei können Parameter wie die Anzahl der Schichten, die Größe der Schichten sowie die Dimensionen der Ein- und Ausgabeschichten festgelegt werden \citep{dokumentation_n++}. Nach der Konfiguration des Netzes können Eingabemuster durch Vorwärtspropagation propagiert werden. Die resultierenden Ausgaben können abgerufen werden, und optional kann durch Rückwärtspropagation des Fehlervektors ein Lernprozess des Netzwerks simuliert werden, wobei die Gewichte automatisch angepasst werden. Generierte Netze können in Dateien gespeichert werden, um sie zu einem späteren Zeitpunkt wiederzuverwenden, insbesondere für reproduzierbare Experimente.

In Codeausschnitt \ref{fig:npp_examplenet_code}, welcher eine vereinfachte Form des Beispielnetzes aus der N++-Dokumentation darstellt, wird ein Beispielnetz mit drei Schichten erstellt. Die Eingabeschicht hat dabei zwei Parameter, die Ausgabeschicht drei, und die versteckte Schicht hat vier Parameter. Es ist ersichtlich, dass die N++-Bibliothek das einfache Austauschen von Updatefunktionen unterstützt. In den Zeilen 16 und 17 wird die Updatefunktion dynamisch auf Rückwärtspropagation gesetzt, was ohne großen Aufwand möglich ist \citep{dokumentation_n++}.

\begin{figure}[!ht]
\begin{minted}
[
frame=lines,
framesep=2mm,
baselinestretch=1.2,
fontsize=\footnotesize,
linenos
]{c++}
#include "n++.h"

#define INPUTS 2
#define OUTPUTS 3
#define LAYERS 3

int main() {
    Net net;
    // Schichten des Netzes erstellen und miteinander verbinden
    int layerNodes[LAYERS] = {INPUTS, 4, OUTPUTS};
    net.create_layers(LAYERS, layerNodes);
    net.connect_layers();
    // Gewichte mit Zufallszahlen zwischen 0 und 0,5 initialisieren
    net.init_weights(0, 0.5);
    // Updatefunktion auf Rückwärtspropagation setzen
    float uparams[5] = {0.1, 0.9, 0, 0, 0};
    net.set_update_f(BP, uparams);
}
\end{minted}
\caption{Vereinfachte Form des Beispielnetzes aus der N++-Dokumentation}
\label{fig:npp_examplenet_code}
\end{figure}

\section{Bestehender Code}
\label{sec:Bestehender_Code_Brening}
\sectionmark{Bestehender Code}

Als bestehender Code wird ein Experiment aus der Bachelorarbeit von \citep{thesis_Artur_Brening}

\section{Voraussetzungen}
\label{sec:Voraussetzungen_Parallelisierung}
\sectionmark{Voraussetzungen}
Für eine erfolgreiche Parallelisierung des Anwendercodes ist eine umfassende Analyse und Modifikation desselben unerlässlich. Dieser Abschnitt diskutiert die grundlegenden Voraussetzungen, die vor der Implementierung von Parallelisierungsstrategien berücksichtigt werden müssen. In erster Linie erfordert die Parallelisierung die Identifizierung und Beseitigung von Abhängigkeiten innerhalb des Algorithmus sowie die Anpassung der Implementierung, um die Effizienz und Skalierbarkeit auf mehreren Prozessoren oder Rechenkernen zu gewährleisten.

\subsection{Entfernung von geteilten Speicherzugriffen}
\label{sec:Entfernung_geteilte_Speicherzugriffe}
Geteilte Speicherzugriffe, häufig realisiert durch globale Variablen im Quellcode, ermöglichen es verschiedenen Teilen eines Programms, auf dieselben Daten zuzugreifen. Während dies in einer sequenziellen Umgebung funktionieren kann, können Probleme auftreten, wenn versucht wird, solche Konstrukte in einem parallelen Kontext zu verwenden.

Bei der Parallelisierung eines Programms ist es entscheidend, dass verschiedene Threads oder Prozesse unabhängig voneinander arbeiten können, um eine effiziente Ausführung zu gewährleisten. Globale Variablen führen jedoch zu potenziellen Konflikten, da mehrere Threads gleichzeitig auf den gleichen Speicherbereich zugreifen können. Dies kann zu Wettlaufbedingungen, inkonsistenten Zuständen und anderen unerwarteten Verhaltensweisen führen, die die Zuverlässigkeit und Korrektheit des Programms beeinträchtigen.

Um dieses Problem zu lösen, ist es notwendig, die Abhängigkeit von geteilten Speicherzugriffen zu reduzieren. Dies erfolgt durch die Umstrukturierung des Quellcodes, um den Einsatz globaler Variablen zu minimieren oder ganz zu eliminieren. Statt globaler Variablen können lokale Variablen verwendet werden, die nur innerhalb bestimmter Funktionsbereiche gültig sind und somit den Zugriff auf den Speicher einschränken. Darüber hinaus können Datenstrukturen wie Klassen oder Strukturen verwendet werden, um Daten zu kapseln und den Zugriff über definierte Schnittstellen zu ermöglichen.

Bei der Entfernung von geteilten Speicherzugriffen ist es wichtig, auch geeignete Synchronisationsmechanismen einzuführen, um kritische Abschnitte des Codes zu schützen. Dies kann die Verwendung von Mutexen, Semaphoren oder anderen Mechanismen umfassen, um sicherzustellen, dass nur ein Thread gleichzeitig auf bestimmte Ressourcen zugreifen kann, und so potenzielle Wettlaufbedingungen zu vermeiden.

Indem geteilte Speicherzugriffe entfernt und durch geeignete Alternativen ersetzt werden, wird die Grundlage für eine zuverlässige und effiziente Parallelisierung des Programms geschaffen.

\subsection{Verlagerung der zu parallelisierenden Routine}
\label{sec:Verlagerung_parallelisierende_Routine}
Um eine effektive Parallelisierung zu erreichen, ist es von entscheidender Bedeutung, den spezifischen Teil des Codes zu identifizieren, der für die parallele Ausführung geeignet ist. Dieser Prozess erfordert eine sorgfältige Analyse des Quellcodes, um Bereiche zu lokalisieren, die unabhängig voneinander ausgeführt werden können und keine oder nur minimale Abhängigkeiten zu anderen Teilen des Programms aufweisen. Solche Bereiche können typischerweise Schleifen oder Abschnitte sein, die große Mengen von Daten verarbeiten, ohne auf Zwischenergebnisse anderer Bereiche angewiesen zu sein.

Nachdem der geeignete Bereich identifiziert wurde, ist es notwendig, ihn aus dem Hauptcode auszulagern und in eine separate Routine oder Funktion zu überführen. Diese ausgelagerte Routine sollte autonom arbeiten können, ohne auf globale Variablen oder gemeinsam genutzte Ressourcen außerhalb ihres Bereichs zuzugreifen. Durch diese Isolierung können potenzielle Konflikte vermieden und die Parallelisierung erleichtert werden.

Es ist entscheidend sicherzustellen, dass die ausgelagerte Routine keine Abhängigkeiten zu anderen Teilen des Codes hat, um eine effiziente Parallelisierung zu ermöglichen. Hierbei müssen gegebenenfalls erforderliche Parameter übergeben und Rückgabewerte behandelt werden, um eine reibungslose Interaktion mit dem Rest des Programms zu gewährleisten.

Die Verlagerung der zu parallelisierenden Routine ist ein grundlegender Schritt bei der Implementierung von Parallelisierungsstrategien und bildet die Grundlage für eine effiziente und robuste parallele Ausführung des Programms. Durch die Identifizierung und Isolierung geeigneter Bereiche können potenzielle Engpässe reduziert und die Leistung des Programms optimiert werden.

\section{Vorstellung der Implementierung}
\label{sec:Vorstellung_Implementierung}

\subsection{Verwendung von threadsicheren Funktionen}
\label{sec:Verwendung_threadsichere_Funktionen}

\subsection{Entfernung globaler Variablen}
\label{sec:Entfernung_globaler_Variablen}
\chapter{Experimentelle Untersuchungen}
\chaptermark{Experimentelle Untersuchungen}  % short form for headlines on pages
\label{ch:EntwickelteMethode}

\section{Angewendete Methodik}
\subsection{Testumgebung}

Um die Wirksamkeit der Parallelisierung der N++-Bibliothek in C++ zu bewerten, wird ein umfassender Benchmark-Test durchgeführt. Dieser Test umfasst verschiedene Kombinationen von Threads, Anzahl an Datensätzen und verschiedenen Computern mit verschiedenen Prozessorarchitekturen, um die Auswirkungen der Implementierung auf die Leistung der Bibliothek unter unterschiedlichen Bedingungen zu untersuchen.

Das C++ Programm wurde unter Einbindung der N++-Bibliothek auf dem jeweiligen System selbst kompiliert. Dabei wurde die Optimierungsstufe O2 verwendet, welche eine für Produktionssoftware gängige Optimierungsstufe ist. Die O2 Optimierungsstufe wendet fast jede Compileroptimierung an, die die Compiler zu bieten haben. Dabei werden lediglich als sehr unsicher eingestufte Optimierungen ausgelassen. Auf Linux wurde der GCC Compiler (Version 12.2.0-14) und auf MacOS der Clang Compiler (Version 1500.3.9.4) verwendet, um native Binärdateien für die spezifische Prozessorarchitektur zu kompilieren. Das heißt, das Programm wurde nicht unter Emulation sondern vollständig nativ ausgeführt.

Die Tests wird mit unterschiedlichen Thread-Anzahlen ausgeführt, darunter 10, 8, 6, 4 und 2 gleichzeitig laufenden Threads, um den Einfluss der Parallelisierung auf die Ausführungsgeschwindigkeit zu untersuchen. Zusätzlich wird das Programm auch mit einem einzelnen Thread ausgeführt, um einen Vergleich mit der vorausgegangenen Implementierung herstellen zu können. Für jede Thread-Anzahl werden außerdem verschiedene Größen an Datensätzen getestet. Die Größe der Datensätze wird über die Anzahl an Partitionen spezifiziert. Eine Partition bedeutet dabei, dass die gesamte Datenmenge verwendet wird, wohingegen 4 Partitionen bedeuten, dass nur ein Viertel, also 25\% der Datenmenge verwendet werden. Das Programm wird in diesem Test mit 1, 2, 4 und 8 Partitionen getestet, wobei es auf dem langsamsten Computer nur auf 4 und 8 beschränkt wurde.

Für jede Kombination von Threads und Partitionen werden mindestens 5 Testläufe durchgeführt, um robuste Durchschnittswerte zu erhalten und Schwankungen zu minimieren. Gemessen wird die benötigte Zeit für den gesamten Programmdurchlauf in Sekunden. Vor jedem Durchlauf werden die Testgeräte auf einen neutralen Zustand zurückgesetzt, um faire Vergleichsbedingungen sicherzustellen. Dies wird gewährleistet, indem gleiche Seeds für die Zufallszahlgeneratoren verwendet werden, und sichergestellt wird, dass keine anderen Programme laufen.
Das Ergebnis jedes Programmdurchlaufs wird in eine Datei geschrieben, um vergleichen zu können, ob mit verschiedenen Anzahlen von Threads die gleichen Ergebnisse berechnet werden.

Nach Abschluss der Testläufe werden die erzielten Ergebnisse automatisch analysiert und Durchschnittswerte für jede Kombination von Threads und Partitionen berechnet. Diese Durchschnittswerte dienen dazu, die möglichen Schwankungen der Testabläufe auszugleichen, und ein neutraleres Ergebnis zu liefern.

\subsection{Benchmark Script}

Aus 6 verschiedenen Anzahlen an Threads, 4 verschiedenen Größen der Datensätze und 5 Durchläufen pro Kombination ergeben sich 120 einzelne Tests, die pro System ausgeführt werden müssen.
Um diese Arbeit zu erleichtern und einen reproduzierbaren Testprozess zu ermöglichen habe ich ein Skript geschrieben, welches alle Tests nacheinander automatisch ausführt.

Das Skript misst die benötigten Laufzeiten der Durchläufe und schreibt nach jedem Durchlauf die benötigte Zeit in eine Datei. Zusätzlich werden die Zeit und Informationen zu jedem Durchlauf auch in eine CSV Datei geschrieben, um das Auswerten der Benchmarks auf einem System durch nur eine einzige Datei zu ermöglichen.

Als Parameter ist es möglich, die maximale Anzahl an Threads festzulegen. So ist es beispielsweise auf einem Raspberry Pi sinnvoll, das Programm nur mit maximal 4 Threads zu testen, da er nur über 4 Threads verfügt.

Das Skript ist simpel und wurde in Bash geschrieben, was die Portabilität zwischen Linux und MacOS gewährleistet. Dabei wurde jedoch das Programm bc verwendet, welches auf Linux standardmäßig nicht vorinstalliert ist.

\begin{figure}[H]
\begin{minted}
[
frame=lines,
framesep=2mm,
baselinestretch=1.2,
fontsize=\footnotesize,
linenos
]{bash}
#!/bin/bash
ATTEMPTS=5
PARTITIONS=(8 4 2 1) # Verschiedene Partitionen zum Testen
THREADS=(10 8 6 4 2 1) # Verschiedene Thread Anzahlen zum Testen
AVAILABLE_THREADS=10 # Maximal verfügbare Anzahl an Threads für aktuelles System

function run_attempt {
    # Hier wird ein Durchlauf durchgeführt und getestet
}

# Test für jede mögliche Kombination durchführen
for partition in ${PARTITIONS[@]};do
  for threads in ${THREADS[@]};do
      if ((threads <= AVAILABLE_THREADS));then
        for ((i = 1; i <= ATTEMPTS; i++)); do
          run_attempt "$threads" "$partition" "$i"
        done
      fi
    done
done
\end{minted}
\caption{Vereinfacht dargestelltes Benchmark Skript}
\label{fig:benchmark_script_code}
\end{figure}

\section{Ergebnisse}

\subsection{Apple M1 Pro}

Der M1 Pro Prozessor, der im MacBook Pro 16 Zoll verwendet wird, integriert eine heterogene Mehrkernarchitektur, die auf die Parallelverarbeitung von Aufgaben ausgelegt ist. Der M1 Pro wurde 2021 vorgestellt und ist eine hochskalierte Version des M1 Prozessors. Er verfügt über insgesamt 12 CPU-Kerne, darunter 8 Hochleistungskerne und 4 Effizienzkerne \citep{MacBook_Technische_Daten}. Die Hochleistungskerne sind für rechenintensive Aufgaben konzipiert, während die Effizienzkerne für weniger anspruchsvolle Aufgaben und eine Reduzierung des Energieverbrauchs optimiert sind.

In Bezug auf die Leistung bietet der M1 Pro Prozessor eine sehr beeindruckende Single-Core-Leistung sowie eine bemerkenswerte parallele Verarbeitungsfähigkeit für multithreaded Anwendungen. Die genaue Hauptspeichergröße beträgt 16 GB RAM mit einer Speicherbandbreite von 200 GBit/s \citep{MacBook_Technische_Daten}. Auch die Cache-Größe des Prozessors ist sehr hoch, was die Latenz bei Speicherzugriffen zusätzlich vermindert.

Verwendet wurde MacOS Sonoma 14.5 mit dem Darwin Kernel Version 23.5.0 auf einem MacBook Pro 16 Zoll Laptop.

\begin{figure}[H]
\centering
\includegraphics[width=0.8\textwidth]{../results/plots/m1pro/comp_all_threads.pdf}
\caption{Performance-Benchmark auf Apple M1 Pro: Einfluss von Thread-Anzahlen auf Verarbeitungszeit bei variierenden Datensatzgrößen}
\label{fig:m1pro_benchmark_threads}
\end{figure}

Die Analyse der Benchmark-Ergebnisse auf dem M1 Pro-Prozessor aus Abbildung \ref{fig:m1pro_benchmark_threads} liefert wertvolle Einblicke in die Leistungsfähigkeit seiner Mehrkernarchitektur und der Effektivität von Parallelisierung des Programms. Durch Tests mit 10, 8, 6, 4, 2 und 1 Threads konnte eine nahezu lineare Skalierung der Leistung beobachtet werden, wobei 10 Threads annähernd eine zehnfache Beschleunigung im Vergleich zu einem einzelnen Thread erzielten.

Interessanterweise zeigt sich jedoch eine bemerkenswerte Konvergenz der Leistungskurven bei 6 und 8 Threads. Dies resultiert aus der Verteilung der parallelen Aufgaben auf die verfügbaren Threads. Mit 6 Threads werden zunächst 6 Aufgaben bearbeitet, während 4 Aufgaben verbleiben. Die verbleibenden 4 Aufgaben erfordern ungefähr dieselbe Ausführungszeit wie die 2 zusätzlichen Aufgaben bei Verwendung von 8 Threads. Diese Konvergenz erklärt die nahezu identischen Leistungskurven bei 6 und 8 Threads.
Bei 6 Threads könnten zwar insgesamt 6 Aufgaben bearbeitet werden, jedoch sind beim zweiten Durchlauf 2 Threads im Leerlauf, während die verbleibenden 4 Threads genutzt werden. Bei 8 Threads bleiben im zweiten Durchlauf sogar 6 Threads inaktiv. Diese Effizienzunterschiede führen zu einer vergleichbaren Ausführungszeit für die verbleibenden Aufgaben bei 6 und 8 Threads.

\begin{figure}[htbp!]
\centering
\includegraphics[width=0.8\textwidth]{../results/plots/m1pro/comp_all_partitions.pdf}
\caption{Performance-Benchmark auf Apple M1 Pro: Einfluss von Datensatzgrößen auf Verarbeitungszeit unter variierender Thread-Anzahl}
\label{fig:m1pro_benchmark_partitions}
\end{figure}

Die Konvergenz ist auch in Abbildung \ref{fig:m1pro_benchmark_partitions} ersichtlich, in welcher die Leistung im Bezug auf verschiedene Datensatzgrößen verglichen wird. Dort ist klar erkennbar, das die Linien von 6 zu 8 Threads waagerecht verlaufen, was einem Gleichbleiben der benötigten Zeit für die Berechnung entspricht.

Des Weiteren ist der Overhead der Parallelisierung ein wichtiger Aspekt, der bei der Interpretation der Ergebnisse berücksichtigt werden muss. Trotz der Parallelisierung von Aufgaben bleibt der Overhead auf dem M1 Pro-Prozessor in diesem spezifischen Kontext gering, jedoch definitiv nicht vernachlässigbar. Die Skalierung der Leistung bleibt hoch. Für die gesamte Datensatzgröße (1 Partition) lag die benötigte Zeit bei einem Thread bei 948,6 Sekunden, während mit 10 Threads eine Zeit von 129,4 Sekunden erzielt wurde. Bei einer theoretisch perfekten Skalierung der Parallelisierung wären bei 10 Threads 94,86 Sekunden zu erwarten, jedoch läuft beispielsweise die Auswertung der Ergebnisse nach dem Programmablauf immer sequenziell, und benötigt somit gleich viel Zeit unabhängig von der verwendeten Thread Anzahl. Des Weiteren ist der bereits angesprochene Overhead der Parallelisierung ein Faktor. Dieser kann teilweise auf das Betriebssystem zurückzuführen sein, das Ressourcen für die Verwaltung und Koordination der Threads bereitstellen muss, was zu zusätzlicher Latenz führen kann, insbesondere wenn die CPU bereits stark ausgelastet ist. Bei voller CPU-Auslastung können Temperaturthrottling-Mechanismen eingreifen, um die Betriebstemperatur der CPU zu regulieren, was zu vorübergehenden Leistungseinbußen führen kann, da die Taktfrequenz reduziert wird, um die Temperaturen im sicheren Bereich zu halten. Ein weiterer Faktor ist die begrenzte Anzahl von Performance-Kernen auf dem M1 Pro Prozessor, die dazu führen kann, dass bei einer höheren Thread-Anzahl als verfügbaren Performance-Kernen nicht alle Threads auf gleich schnellen Kernen laufen können.

\subsection{AMD Ryzen 3600XT}
Der AMD Ryzen 5 3600XT ist ein Prozessor aus der Ryzen 3000-Serie von AMD, der im Jahr 2020 eingeführt wurde. Er verfügt über insgesamt 6 CPU-Kerne und 12 Threads auf Basis der Zen 2-Architektur, welche auch im Servermarkt verbreitet ist. Es handelt sich um einen Desktopprozessor mit einer maximalen Leistungsaufnahme von 95 Watt \citep{Ryzen_Technische_Daten}.

Der Prozessor unterstützt DDR4-RAM mit unübertakteten Geschwindigkeiten von bis zu 3200 MHz \citep{Ryzen_Technische_Daten}. Die genaue Speicherbandbreite und Größe hängt von der verwendeten RAM-Konfiguration ab, da Desktop Prozessoren modular in verschiedene Systeme eingebaut werden können.

In Bezug auf den Cache verfügt der Ryzen 5 3600XT über 32KB L1-Cache, 512KB L2-Cache und 32MB L3-Cache \citep{Ryzen_Technische_Daten}. In der Konfiguration des Testcomputers sind 16GB DDR4 Speicher mit 3600 MHz verbaut, und der Prozessor wird mit einer Wasserkühlung gekühlt, was für eine gleichmäßige und robuste Kühlung sorgt.

\begin{figure}[H]
\centering
\includegraphics[width=0.8\textwidth]{../results/plots/3600xt/comp_all_threads.pdf}
\caption{Performance-Benchmark auf Ryzen 5 3600XT: Einfluss von Thread-Anzahlen auf Verarbeitungszeit bei variierenden Datensatzgrößen}
\label{fig:ryzen_benchmark_threads}
\end{figure}

\begin{figure}[H]
\centering
\includegraphics[width=0.8\textwidth]{../results/plots/3600xt/comp_all_partitions.pdf}
\caption{Performance-Benchmark auf Ryzen 5 3600X: Einfluss von Datensatzgrößen auf Verarbeitungszeit unter variierender Thread-Anzahl}
\label{fig:ryzen_benchmark_partitions}
\end{figure}

\subsection{Raspberry Pi 3}
Der Raspberry Pi 3 ist ein Single-Board-Computer, der von der Raspberry Pi Foundation entwickelt wurde. Ein Single-Board-Computer ist eine vollständige Computerplatine, die alle erforderlichen Komponenten wie Prozessor, Speicher, Ein-/Ausgabeanschlüsse und Stromversorgung auf einer einzigen Platine vereint. Der Raspberry Pi 3 wurde 2016 veröffentlicht und basiert auf einem ARM Cortex-A53 Quad-Core-Prozessor mit einer Taktfrequenz von 1,2 GHz und 1 GB LPDDR2-RAM \citep{RaspberryPi_Technische_Daten}.

Im Vergleich zu anderen Prozessoren ist der Raspberry Pi 3 natürlich langsamer. Seine Spezifikationen bieten eine grundlegende Leistung für einfache Computing-Aufgaben und den Betrieb von IoT (Internet der Dinge)-Anwendungen.

Der Raspberry Pi 3 ist aufgrund seiner geringen Größe, seines geringen Stromverbrauchs und seiner vielfältigen Einsatzmöglichkeiten beliebt. Er wird häufig in Bildungsprojekten, DIY (Do It Yourself)-Projekten, Heimautomatisierungssystemen und als kostengünstige Entwicklungsumgebung für Softwareentwickler eingesetzt. Trotz seiner begrenzten Leistungsfähigkeit erfüllt der Raspberry Pi 3 wichtige Funktionen in verschiedenen Anwendungsbereichen aufgrund seiner Kompaktheit und seines erschwinglichen Preises.

Aufgrund der begrenzten Leistung des Raspberry Pi 3 und der damit verbundenen Laufzeiten konnte der Testablauf nicht vollständig durchgeführt werden. Es wurden maximal 4 Threads getestet, und die Partitionen wurden auf 8 und 4 begrenzt.

\begin{figure}[H]
\centering
\includegraphics[width=0.8\textwidth]{../results/plots/raspberrypi3/comp_all_threads.pdf}
\caption{Ergebnisse der Leistungstests verglichen nach Thread Anzahl}
\label{fig:raspi_benchmark_threads}
\end{figure}

\begin{figure}[htbp!]
\centering
\includegraphics[width=0.8\textwidth]{../results/plots/raspberrypi3/comp_all_partitions.pdf}
\caption{Ergebnisse der Leistungstests verglichen nach Datensatzgröße}
\label{fig:raspi_benchmark_partitions}
\end{figure}

\subsection{Cloud Server}

\begin{figure}[H]
\centering
\includegraphics[width=0.8\textwidth]{../results/plots/vps/comp_all_threads.pdf}
\caption{Ergebnisse der Leistungstests verglichen nach Thread Anzahl}
\label{fig:vps_benchmark_threads}
\end{figure}

\begin{figure}[htbp!]
\centering
\includegraphics[width=0.8\textwidth]{../results/plots/vps/comp_all_partitions.pdf}
\caption{Ergebnisse der Leistungstests verglichen nach Datensatzgröße}
\label{fig:vps_benchmark_partitions}
\end{figure}

\section{Auswertung}

\newpage

\chapter{Fazit und Ausblick}
\label{ch:Zusammenfassung}

Der Code der Vorarbeit wurde parallelisiert und die n++-Bibliothek wurde threadsicher umgestaltet. Durch ausführliche Leistungstests konnte die Effizienz der Parallelisierung hinsichtlich der benötigten Laufzeit untersucht werden. Eine deutliche Verbesserung der benötigten Laufzeiten konnte erzielt und aufgezeigt werden. Es wurde bestätigt, dass sich die erzielten Verbesserungen nicht auf lediglich ein System beschränken, sondern auf mehreren Prozessorarchitekturen und Betriebssystemen realisiert worden sind.

\begin{itemize}
    \item Erweiterung der n++-Bibliothek: Die n++-Bibliothek könnte um weitere Funktionen zur Parallelisierung erweitert werden. Dies könnte zum Beispiel die Parallelisierung von internen Codeabschnitten oder die Bereitstellung von Hilfsfunktionen zur Optimierung der Thread-Zuordnung umfassen.
    \item Untersuchung weiterer Parallelisierungsstrategien: Neben der Parallelisierung auf CPU-Ebene könnten auch andere Parallelisierungsstrategien untersucht werden, z. B. die Parallelisierung auf GPU-Ebene oder die Verwendung von verteilten Systemen.
\end{itemize}

\begin{appendix}
  \input{TeXFiles/090_Anhang}
  \clearpage
\end{appendix}

%Falls man die Überschrift des Literaturverzeichnisses
%aendern moechte, geht das durch Verwendung der folgenden Zeile:
%\renewcommand*{\refname}{Literaturverzeichnis}

%\bibliographystyle{plain} % Nummern in eckigen Klammern
%%\bibliographystyle{alpha} % Anfangsbuchstaben Erstautor und Jahr

% Wenn man folgenden Stil verwenden will, dann muss
% \usepackage{harvard} vor \begin{document} aktiviert werden:
%\bibliographystyle{agsm} % (Autoren in runden Klammern)
%\newpage
\clearpage


\phantomsection % Da sonst falsche Verlinkung im Inhaltsverzeichnis
\addcontentsline{toc}{chapter}{Abbildungsverzeichnis}
\listoffigures
\printbibliography[heading=bibintoc]    % Ohne Kapitelnummer /-buchstabe
\end{document}
